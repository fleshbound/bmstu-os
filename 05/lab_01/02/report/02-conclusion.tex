\chapter*{ВЫВОД}
\addcontentsline{toc}{chapter}{ВЫВОД}

Функции обработчика прерывания от системного таймера для операционных систем семейства UNIX и для семейства Windows схожи, так как они являются системами разделения времени.
Схожие задачи обработчика прерывания от системного таймера:
\begin{itemize}
	\item декремент кванта (текущего процесса в UNIX или потока в Windows);
	\item инициализация (но не выполнение) отложенных действий, которые относятся к работе планировщика (например, пересчет приоритетов);
	\item декремент счетчиков тиков (таймеров, часов, счетчиков времени отложенных действий, будильников реального времени).
\end{itemize}

Пересчет динамических приоритетов осуществляется только для пользовательских процессов, чтобы избежать бесконечного откладывания.
ОС обоих семейств UNIX и Windows --- это системы разделения времени с динамическими приоритетами и вытеснением.

В UNIX приоритет пользовательского процесса (процесса в режиме задачи) может динамически пересчитываться, в зависимости от трех факторов. 
Приоритеты ядра --- фиксированные величины.

В Windows при создании процесса ему назначается базовый приоритет, относительно базового приоритета процесса потоку назначается относительный приоритет, таким образом, у потока нет своего приоритета.
Приоритет потока пользовательского процесса может быть динамически пересчитан.