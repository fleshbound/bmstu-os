\chapter{Пересчет динамических приоритетов}

В ОС семейств UNIX/Linux и Windows динамически пересчитываться могут только приоритеты пользовательских процессов.

\section{UNIX/Linux}

Приоритет процесса --- это целое числом, находящееся в диапазоне от 0 до 127.
Чем меньше число, тем выше приоритет процесса.
Приоритеты ядра находятся в диапазоне от 0 до 49, они зарезервированы.
Приоритеты прикладных задач в диапазоне от 50 до 127.

Дескриптор процесса \textbf{struct proc} содержит следующие поля, которые относятся к приоритету процесса:
\begin{itemize}
	\item \textbf{p\_runpri} --- приоритет выполнения процесса в текущий момент времени;
	\item \textbf{p\_slppri} --- приоритет состояния ожидания;
	\item \textbf{p\_usrpri} --- приоритет режима задачи;
	\item \textbf{p\_cpu} --- результат последнего измерения использования процессора;
	\item \textbf{p\_nice} --- фактор <<любезности>>, который устанавливается пользователем.
\end{itemize}

Планировщик использует p\_runpri для принятия решения о том, какой процесс направить на выполнение, а именно для хранения временного приоритета для выполнения в режиме ядра.

Поле p\_usrpri используется для хранения приоритета, который будет назначен процессу при возврате в режим задачи из состояния блокировки.
У событий или объектов ядра, на которых может быть блокирован процесс, определён приоритет сна. 
Приоритет сна являтся величиной, определяемой для ядра, и потому лежит в диапазоне 0--49.
В таблице \ref{tab:bsd_2} приведены значения приоритетов сна для некоторых событий в системе 4.3BSD.

Можно выделить следующие особые ситуации, связанные с изменением
полей p\_usrpri, p\_runpri:
\begin{itemize}
	\item когда процесс находится в режиме задачи, то его значения полей p\_usrpri и p\_runpri равны;
	\item  когда процесс просыпается после блокирования, то есть происходит его постановка в очередь готовых процессов, его приоритету p\_runpri присваивается значение приоритета сна события или ресурса, на котором он был блокирован, чтобы дать процессу предпочтение для выполнения в режиме ядра;
	\item когда процесс завершил выполнение системного вызова и находится в состоянии возврата в режим задачи, его приоритет p\_runpri сбрасывается обратно в значение текущего приоритета в режиме задачи p\_usrpri.
\end{itemize}

\begin{table}[h]
	\caption{Системные приоритеты сна}
	\label{tab:bsd_2}
	\begin{center}
		\begin{tabular}{ |c|c|p{9cm}|  }
			\hline
			\textbf{4.3BSD UNIX} & \textbf{SCO UNIX} & \textbf{Событие} \\
			\hline
			\texttt{0} & 95 & Ожидание загрузки в память сегмента/страницы (свопинг, страничное замещение) \\
			\hline
			\texttt{10} & 88 & Ожидание индексного дескриптора \\
			\hline
			\texttt{20} & 81 & Ожидание ввода-вывода \\
			\hline
			\texttt{30} & 80 & Ожидание буфера \\
			\hline
			\texttt{-} & 75 & Ожидание терминального ввода \\
			\hline
			\texttt{-} & 74 & Ожидание терминального вывода \\
			\hline
			\texttt{-} & 73 & Ожидание завершения выполнения \\
			\hline
			\texttt{40} & 66 & Ожидание события --- низкоприоритетное состояние сна \\
			\hline
		\end{tabular}
	\end{center}
\end{table}

\begin{table}[!h]
	\caption{Таблица приоритетов сна в ОС 4.3BSD}
	\label{tab:bsd}
	\begin{center}
		\begin{tabular}{ |c|c|c|  }
			\hline
			\textbf{Приоритет} & \textbf{Значение} & \textbf{Описание} \\
			\hline
			\texttt{PSWP} & 0 & Свопинг \\
			\hline
			\texttt{PSWP + 1} & 1 & Страничный демон \\
			\hline
			\texttt{PSWP + 1/2/4} & 1/2/4 & Другие действия по обработке памяти \\
			\hline
			\texttt{PINOD} & 10 & Ожидание освобождения inode \\
			\hline
			\texttt{PRIBIO} & 20 & Ожидание дискового ввода-вывода \\
			\hline
			\texttt{PZERO} & 25 & Базовый приоритет \\
			\hline
			\texttt{TTIPRI} & 28 & Ожидание ввода с терминала \\
			\hline
			\texttt{TTOPRI} & 29 & Ожидание вывода с терминала \\
			\hline 
			\texttt{PWAIT} & 30 & Ожидание завершения процесса-потомка \\
			\hline
			\texttt{PLOCK} & 35 & Консультативное ожидание блок.
			ресурса \\
			\hline
			\texttt{PSLEP} & 40 & Ожидание сигнала \\
			\hline
		\end{tabular}
	\end{center}
\end{table}

Изменение приоритета в режиме задачи зависит от двух факторов: 
\begin{itemize}
	\item фактора <<любезности>> (\textbf{p\_nice});
	\item последней измеренной величины использования процессора (\textbf{p\_estcpu}).
\end{itemize}

Фактор <<любезности>> --- целое число в диапазоне от 0 до 39 (по умолчанию 20).
Чем меньше значение фактора <<любезности>>, тем выше приоритет процесса.

Каждую секунду ядро системы инициализирует отложенный вызов процедуры \textbf{schedcpu()}, которая уменьшает значение \textbf{p\_runpri} каждого процесса исходя из фактора <<полураспада>> (decay factor). В системе 4.3BSD фактор <<полураспада>> рассчитывается по формуле \ref{eq:ref1}:
\begin{equation}
	\label{eq:ref1}
	decay = \frac{2 \cdot load\_average}{2 \cdot load\_average + 1} ,
\end{equation}
где \textit{load\_average} --- среднее количество процессов, находящихся в состоянии готовности к выполнению (за последнюю секунду).

Также процедура \textbf{schedcpu()} пересчитывает приоритеты для режима задачи всех процессов по формуле \ref{eq:ref2}:
\begin{equation}\label{eq:ref2}
	p\_usrpri=PUSER+\frac{p\_estcpu}{4}+2\cdot p_nice,
\end{equation}
где \textbf{PUSER} --- базовый приоритет в режиме задачи, равный 50.

Если процесс в последний раз использовал большое количество процессорного времени, то его \textbf{р\_estсрu} будет увеличен.
Это приведет к росту значения \textbf{p\_usrpri}, из чего последует понижение приоритета.
Чем дольше процесс простаивает в очереди на выполнение, тем больше фактор <<полураспада>> уменьшает его \textbf{р\_estсрu}, что приводит к повышению его приоритета.

Такая схема предотвращает бесконечное откладывание процессов. 
Применение данной схемы предпочтительно процессам, осуществляющим много операций ввода-вывода, в противоположность процессам, производящим много вычислений.
То есть динамический пересчет приоритетов процессов в режиме задачи позволяет избежать бесконечного откладывания.

\section{Windows}

В Windows процессу при создании назначается базовый приоритет.
Процесс по умолчанию наследует свой базовый приоритет у того процесса, который его создал.
Относительно базового приоритета процесса потоку назначается относительный приоритет.

В Windows реализуется приоритетная вытесняющая система планирования, при которой всегда выполняется хотя бы один готовый поток с самым высоким приоритетом.

В Windows используется 32 уровня приоритета, которые описываются целыми числами от 0 до 31:
\begin{itemize}
	\item 0 --- зарезервирован для процесса обнуления страниц;
	\item от 0 до 15 --- динамически изменяющиеся уровни;
	\item от 16 до 31 --- уровни реального времени;
	\item 31 --- наивысший приоритет.
\end{itemize}

Уровни приоритета потоков назначаются с двух позиций: Windows API и ядра Windows.

Сначала Windows API систематизирует процессы по классам приоритетов:
\begin{itemize}
	\item реального времени (Real-time, 4);
	\item высокий (High, 3);
	\item выше обычного (Above Normal, 6);
	\item обычный (Normal, 2);
	\item ниже обычного (Below Normal, 5);
	\item уровень простоя (Idle, 1).
\end{itemize}

Затем назначается относительный приоритет отдельных потоков в рамках процессов:
\begin{itemize}
	\item критичный по времени (Time-critical, 15);
	\item наивысший (Highest, 2);
	\item выше обычного (Above-normal, 1);
	\item обычный (Normal, 0);
	\item ниже обычного (Below-normal, –1);
	\item самый низший (Lowest, –2);
	\item уровень простоя (Idle, –15)
\end{itemize}

В таблице \ref{tab:prioritet} показано соответствие между приоритетами Windows API и ядра системы.

\begin{table}[!h]
	\caption{Соответствие между приоритетами Windows API и ядра Windows}
	\begin{center}
		\begin{tabular}{|c|c|c|c|c|c|c|}
			\hline
			Класс приоритета & Real-time & High & Above &
			Normal & Below Normal & Idle \\ \hline
			Time Critical & 31 & 15 & 15 & 15 & 15 & 15 \\ \hline
			Highest & 26 & 15 & 12 & 10 & 8 & 6 \\ \hline
			Above Normal & 25 & 14 & 11 & 9 & 7 & 5 \\ \hline
			Normal & 24 & 13 & 10 & 8 & 6 & 4 \\ \hline
			Below Normal & 23 & 12 & 9 & 7 & 5 & 3 \\ \hline
			Lowest & 22 & 11 & 8 & 6 & 4 & 2 \\ \hline
			Idle & 16 & 1 & 1 & 1 & 1 & 1 \\ \hline
		\end{tabular}
	\end{center}
	\label{tab:prioritet}
\end{table}

Планировщик повышает текущий приоритет потока в динамическом диапазоне (от 1 до 15) вследствие следующих причин:
\begin{itemize}
	\item повышение приоритета владельцем блокировки;
	\item повышение приоритета после завершения операции ввода/вывода;
	\item повышение приоритета вследствие ввода из пользовательского интерфейса;
	\item повышение приоритета вследствие длительного ожидания ресурса исполняющей системы;
	\item повышение вследствие ожидания объекта ядра (семафора, мьютекса, объекта <<событие>>);
	\item повышение приоритета в случае, когда поток, готовый к выполнению, не был запущен в течение длительного времени;
	\item повышение приоритета проигрывания мультимедиа службой планировщика \textbf{MMCSS} (таблица \ref{tab:category}).
\end{itemize}

Рекомендуемое повышение значения приоритета кода режима ядра, вызывающего такие функции, как KeReleaseMutex, KeSetEvent и KeReleaseSemaphore, является 1.

\begin{table}[!h]
	\caption{Категории планирования}
	\begin{center}
		\begin{tabular}{|p{40mm}|p{30mm}|p{80mm}|}
			\hline
			\textbf{Категория} & \textbf{Приоритет} & \textbf{Описание} \\
			\hline
			High (Высокая) & 23-26 & Потоки профессионального аудио (Pro
			Audio), запущенные с приоритетом выше, чем у других потоков на системе, за
			исключением критических системных потоков \\
			\hline
			Medium (Средняя) & 16-22 & Потоки, являющиеся частью приложений
			первого плана, например, Windows Media Player \\
			\hline
			Low (Низкая) & 8-15 & Все остальные потоки, не являющиеся частью
			предыдущих категорий \\
			\hline
			Exhausted (Исчерпавших потоков) & 1-7 & Потоки, исчерпавшие свою
			долю времени центрального процессора, выполнение которых продолжиться, только
			если не будут готовы к выполнению другие потоки с более высоким уровнем
			приоритета \\
			\hline
		\end{tabular}
	\end{center}
	\label{tab:category}
\end{table}

\begin{table}[!h]
	\caption{Рекомендуемые значения повышения приоритета}
	\begin{center}
		\begin{tabular}{|p{100mm}|l|}
			\hline
			\textbf{Устройство} & \textbf{Приращение} \\\hline
			Диск, CD-ROM, параллельный порт, видео & 1 \\ \hline
			Сеть, почтовый ящик, именованный канал, последовательный порт & 2 \\ \hline
			Клавиатура, мышь & 6 \\ \hline
			Звуковая плата & 8 \\ \hline
		\end{tabular}
	\end{center}
	\label{tab:input-output}
\end{table}