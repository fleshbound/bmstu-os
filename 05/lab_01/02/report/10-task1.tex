\chapter{Функции обработчика прерывания от системного таймера в системах разделения времени}

\section{UNIX/Linux}

Действия по тику:

\begin{itemize}
	\item инкремент счетчика реального времени;
	\item декремент кванта текущего потока;
	\item декремент счетчиков времени до отправки на выполнение отложенных вызовов (если счетчик достиг нуля, происходит установка флага обработчика отложенных вызовов);
	\item инкремент счетчика процессорного времени, полученного процессом в режиме задачи и в режиме ядра.
\end{itemize}

Действия по главному тику:

\begin{itemize}
	\item инициализация работы планировщика;
	\item пробуждение системных процессов $swapper$, $pagedaemon$;
	\item декремент счетчиков времени до отправки одного из следующих сигналов:
	\begin{enumerate}
		\item $SIGALRM$ --- сигнал, посылаемый процессу по истечении времени, которое предварительно задано функцией $alarm()$;
		\item $SIGPROF$ --- сигнал, посылаемый процессу по истечении времени, которое задано в таймере профилирования;
		\item $SIGVTALRM$ --- сигнал, посылаемый процессу по истечении времени, которое задано в <<виртуальном>> таймере.
	\end{enumerate}
\end{itemize}

Действие по кванту:

\begin{itemize}
	\item посылка сигнала $SIGXCPU$ текущему процессу, если он превысил выделенный ему квант использования процессора.
\end{itemize}

\section{Windows}

Действия по тику:

\begin{itemize}
	\item
\end{itemize}

Действия по главному тику:

\begin{itemize}
	\item
\end{itemize}

Действия по кванту:

\begin{itemize}
	\item
\end{itemize}